\chapter{Soft Regime-Switching Approach}

Building upon the hard regime detection framework, this chapter presents a soft regime-switching approach for volatility forecasting and VaR estimation. Rather than training new models, this methodology leverages the established hard regime GARCH-LSTM models while introducing probabilistic regime assignments to capture the gradual nature of market transitions.

\section{Motivation and Approach}

The hard regime detection validation revealed that K-means with Gaussian kernel transformation achieved the best performance among nine tested methods, with a silhouette score of 0.138. However, the hard regime approach assumes instantaneous transitions between market states, which may not capture the gradual nature of regime changes in financial markets.

Our soft regime approach addresses this limitation by:
\begin{enumerate}
    \item Converting hard K-means assignments to soft probabilities using distance-based weighting
    \item Applying probability-weighted averaging to combine regime-specific forecasts
    \item Leveraging existing hard regime GARCH-LSTM models without retraining
    \item Implementing empirical calibration to ensure realistic VaR violation rates
\end{enumerate}

\section{Soft Regime Probability Generation}

\subsection{From Hard Clusters to Soft Probabilities}

Given the validated K-means clustering with 5 regimes (excluding regime 1 due to insufficient observations), we convert hard assignments to soft probabilities using the same clusters and centroids established in the validation process.

For observation $\mathbf{x}_t$ in the standardized feature space, the Euclidean distance to each regime centroid is:
\begin{align}
d_{k,t} = \|\mathbf{x}_t - \boldsymbol{\mu}_k\|_2, \quad k \in \{0,2,3,4\}
\end{align}

The soft probabilities are computed using a Gaussian kernel with bandwidth parameter $\sigma=1.0$ (as determined optimal in validation):
\begin{align}
P(\text{regime}=k \mid \mathbf{x}_t) = \frac{\exp(-d_{k,t}^2 / (2\sigma^2))}{\sum_{j \in \{0,2,3,4\}} \exp(-d_{j,t}^2 / (2\sigma^2))}
\end{align}

This Gaussian kernel transformation was selected because it achieved:
\begin{itemize}
    \item High confidence assignments: 94.1\% of observations with $P_{\max} > 0.8$
    \item Low entropy: Average entropy of 0.151 (vs. maximum possible 1.61)
    \item Smooth transitions: Preserves continuity while maintaining clear regime identification
\end{itemize}

\subsection{Properties of Soft Regime Probabilities}

The generated soft probabilities exhibit several desirable properties:
\begin{enumerate}
    \item \textbf{Normalization:} $\sum_{k} P(\text{regime}=k \mid \mathbf{x}_t) = 1$ for all observations
    \item \textbf{Continuity:} Smooth transitions between regime assignments over time
    \item \textbf{Boundary sensitivity:} Observations near regime boundaries receive mixed probabilities
    \item \textbf{Temporal stability:} Gradual probability evolution reduces forecast instability
\end{enumerate}

\section{Soft Probability-Weighted Forecasting Framework}

\subsection{Leveraging Pre-trained GARCH-LSTM Models}

Rather than training new regime-specific models, our approach utilizes the previously trained hard regime GARCH-LSTM models. These models, trained on regime-specific data subsets, capture the distinct volatility dynamics of each market state.

For each viable regime $k \in \{0,2,3,4\}$ and asset $i \in \{\text{SPX, RTY, NDX}\}$, we have pre-trained GARCH-LSTM models that provide volatility forecasts:
\begin{align}
\hat{\sigma}^2_{k,i,t+1|t} = f_k(\mathbf{X}_{k,i,t}; \theta_{k,i})
\end{align}

where $f_k(\cdot)$ represents the trained GARCH-LSTM model for regime $k$, and $\theta_{k,i}$ are the estimated parameters.

\subsection{Probability-Weighted Forecast Combination}

The soft regime approach combines individual regime forecasts using the soft probabilities as weights:
\begin{align}
\hat{\sigma}^2_{i,t+1|t}^{\text{soft}} = \sum_{k \in \{0,2,3,4\}} P(\text{regime}=k \mid \mathbf{x}_t) \cdot \hat{\sigma}^2_{k,i,t+1|t}
\end{align}

This probability-weighted averaging approach ensures:
\begin{itemize}
    \item \textbf{Smoothness:} Gradual transitions between model predictions
    \item \textbf{Efficiency:} No retraining required, leveraging existing model infrastructure  
    \item \textbf{Robustness:} Multiple models contribute to final forecast, reducing single-model risk
    \item \textbf{Interpretability:} Clear regime contributions to final predictions
\end{itemize}

\section{Empirical Calibration and Scaling}

\subsection{Volatility Forecast Calibration}

Initial diagnostic analysis revealed that the GARCH-LSTM models predicted volatility levels significantly higher than observed market volatility. This systematic overestimation required empirical calibration to achieve realistic VaR violation rates.

The diagnostic evidence showed:
\begin{itemize}
    \item SPX: 20.09× overestimation (predicted 16.3\% vs. actual 0.8\% daily volatility)
    \item RTY: 14.29× overestimation (predicted 18.3\% vs. actual 1.3\% daily volatility)  
    \item NDX: 16.56× overestimation (predicted 19.0\% vs. actual 1.1\% daily volatility)
\end{itemize}

Following established risk management practices \citep{Campbell2008}, we implement asset-specific calibration factors:
\begin{align}
\text{Calibration Factor}_i = \frac{\text{Mean Realized Volatility}_i}{\text{Mean Predicted Volatility}_i}
\end{align}

\subsection{Calibrated Volatility Forecasting}

The final calibrated volatility forecast incorporates both soft probability-weighted averaging and empirical scaling:
\begin{align}
\hat{\sigma}_{i,t+1|t}^{\text{final}} = \sqrt{\frac{\hat{\sigma}^2_{i,t+1|t}^{\text{soft}}}{22}} \times \text{Calibration Factor}_i
\end{align}

where the division by 22 converts from realized variance (RV22) to daily volatility, and the calibration factor corrects for systematic bias.

\section{Filtered Historical Simulation Implementation}

\subsection{Regime-Aware Historical Simulation}

The soft regime approach implements filtered historical simulation (FHS) using the calibrated volatility forecasts. For each forecast date $t$, we:

\begin{enumerate}
    \item Generate soft probability-weighted volatility forecast $\hat{\sigma}_{i,t+1|t}^{\text{final}}$
    \item Extract historical returns for the past 252 trading days
    \item Standardize returns by historical volatility: $r_{\text{std},s} = r_s / \hat{\sigma}_{\text{hist}}$
    \item Scale by forecasted volatility: $r_{\text{sim},s} = r_{\text{std},s} \times \hat{\sigma}_{i,t+1|t}^{\text{final}}$
    \item Calculate VaR as empirical quantiles: $\text{VaR}_\alpha = Q_\alpha(r_{\text{sim}})$
\end{enumerate}

\subsection{Performance Validation}

The soft regime FHS approach achieved realistic violation rates close to theoretical targets:
\begin{itemize}
    \item SPX: 1.03\% (1\% VaR), 5.13\% (5\% VaR)
    \item RTY: 0.53\% (1\% VaR), 3.74\% (5\% VaR)
    \item NDX: 1.28\% (1\% VaR), 6.15\% (5\% VaR)
\end{itemize}

Statistical validation using Kupiec unconditional coverage tests confirmed that all violation rates are statistically consistent with their theoretical targets (all p-values > 0.24).

\section{Comparison with Hard Regime Approach}

\subsection{Volatility Forecasting Performance}

Comparing the soft-weighted forecasts against the best individual hard regime models:
\begin{itemize}
    \item \textbf{SPX:} Soft RMSE 0.0793 vs. Best Hard RMSE 0.0550 (Regime 3)
    \item \textbf{RTY:} Soft RMSE 0.1124 vs. Best Hard RMSE 0.1118 (Regime 0) - Nearly identical
    \item \textbf{NDX:} Soft RMSE 0.1103 vs. Best Hard RMSE 0.0568 (Regime 3)
\end{itemize}

While individual best hard regime models show superior RMSE performance, the soft approach provides more consistent and stable performance across all market conditions.

\subsection{VaR Performance Stability}

The key advantage of the soft regime approach lies in its stability and reliability:
\begin{enumerate}
    \item \textbf{Consistent validation:} All Kupiec tests passed (p > 0.24)
    \item \textbf{No regime-dependent failures:} Unlike hard regimes, no systematic breakdowns
    \item \textbf{Smooth transitions:} Eliminates discontinuities from regime switching
    \item \textbf{Regulatory compliance:} Reliable violation rates suitable for risk management
\end{enumerate}

\section{Methodology Summary}

This soft regime-switching approach successfully addresses the limitations of hard regime assignment while leveraging the established GARCH-LSTM infrastructure. The key innovations include:

\begin{enumerate}
    \item \textbf{Validated soft probability generation} using Gaussian kernel transformation with optimal bandwidth $\sigma=1.0$
    \item \textbf{Dynamic model averaging} that combines pre-trained regime-specific models without retraining
    \item \textbf{Empirical calibration framework} that ensures realistic VaR performance through systematic bias correction
    \item \textbf{Stable VaR estimation} that passes statistical validation tests across all assets
\end{enumerate}

The approach demonstrates that soft regime-switching can provide reliable risk management tools while maintaining the interpretability and economic intuition of regime-based models. The methodology offers a practical solution for financial institutions requiring robust, validated VaR models for regulatory and internal risk management purposes.